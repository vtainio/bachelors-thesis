% --------------------------------------------------------------------

\section{Introduction}

Communicating with each other using technologies, such as Bluetooth, is becoming ever more popular. The new emerging technologies enable us to create new ways of establishing communication between total strangers with similar interests. This thesis describes how these technologies can be used to create social interaction between strangers and therefore increase their well-being and performance during sports.

Familiar strangers is a concept first introduced by the psychologist Stanley Milgram in 1972 in his essay \citep{milgram1992}. We often come across to the same strangers while doing sports, but do not interact with them. These people, that you have met frequently but never interacted with, are the ones called familiar strangers. \citep{familiarStranger}. Familiar strangers as a concept isn't limited to sports, but targeting the research to people who have similar interests (sports) by definition makes monitoring of their behavior simpler. While social networking between strangers has been research before, this concept of social interaction between familiar strangers in sports, is quite new in the field.

Methods used in this paper to research this problems are:
\begin{itemize}
\item  Literature review.
\item Conducting interviews.
\item Creating a prototype application for research data.
\end{itemize}

This paper presents a prototype Android application that will log strangers passing by. When you come across to a strangers enough times, the application will suggest communication with the stranger. With the prototype, you can view where and how many times you have encountered that person and what are they interested in. Interesting questions related to this prototype application are, whether users are able to establish communication based on similar interest and similar real-life habits (sports routes and times) and also how much information users are willing to share to total strangers. Data gathered form this prototype application can later be used to verify assumptions about the users behavior and to learn new information. The prototype application takes privacy seriously and is quite conservative about sharing information. The level of privacy can later then be modified based on feedback from the users.

The interviews are composed from open-ended questions where the goal is more to find new information rather than just to validate previous assumptions. The interviews are extensive and performed only for a handful of possible end users of the application. No survey's were conducted for this thesis.

\section{Related work}

\cite{socialAdHoc} studied ad hoc social networking. 90\% of the participants in their study would use regularly a service which would help introduce nearby strangers to each other. Serendipity, the application created for their research, is a mobile match-making system which alerts users when someone with similar interests comes into proximity.

Previous research suggests that doing sports in a group or together with a friend results in increased performance. Therefore, finding strangers with similar level of fitness to do sports with would result in performance increase for the users. However, finding people to do sports with can be an daunting task especially for people who have just moved to a new city or a country.

X studied the use of an  ad-hoc based social networking application where one of the results was that people were interested in getting to know strangers via mobile applications. The study also showed that the users weren't concerned about releasing private information in the application.

% --------------------------------------------------------------------

% --------------------------------------------------------------------

\section{Introduction}

Communicating with each other using technologies, such as Bluetooth, is becoming ever more popular in the field. Both old and new emerging technologies enable us to create new ways of establishing communication between total strangers with similar interests. This thesis describes how these technologies can be used to create social interaction between strangers and therefore increase their well-being of people and their performance during sports.

Familiar strangers is a concept first introduced by the psychologist Stanley Milgram in 1972 in his essay \citep{milgram1992}. We often come across to the same strangers while doing sports, but do not interact with them. These people, that you have met frequently but never interacted with, are called familiar strangers. \citep{familiarStranger}. Familiar strangers as a concept isn't limited to sports, but targeting the research to people who have similar interests (sports) by definition makes monitoring of their behavior simpler. While social networking between strangers has been research before, this concept of social interaction between familiar strangers in sports, is new in the field.

Methods used in this paper to research this problems are:
\begin{itemize}
\item  Literature review.
\item Conducting interviews.
\item Creating a prototype application for research data.
\end{itemize}

This paper presents a prototype Android application that will log the times strangers passing by you. When you come across to a strangers enough times, the application will suggest communication with the stranger. With the prototype, you can view where and how many times you have encountered that person and what are they interested in. Interesting questions related to this prototype application are, whether users are willing to establish communication based on similar interest and similar real-life habits (sports routes and times) and also how much information users are willing to share to total strangers. Data gathered form this prototype application can later be used to verify assumptions about the users behavior and to learn new information. The prototype application takes privacy seriously and is quite conservative about sharing information. The level of privacy can later then be modified based on feedback from the users.

The interviews were composed from open-ended questions where the goal was more to find new information rather than just to validate previous assumptions. The interviews were extensive and performed only for a handful of possible end users of the application. No survey's were conducted for this thesis.

\section{Related work}

This section presents related works from two perspectives: the social interaction perspective and the perspective of doing sports. The design of the prototype application relies on results from both of these perspectives.

\subsection{Social networking}
\cite{socialAdHoc} studied ad hoc social networking with a social networking system called TWIN. In a survey conducted after the study, the method for approaching unfamiliar persons was one of the highest rated features of the system. \cite{mobileMatchmaking} conducted a survey where 90\% of the participants stated that they would use regularly a service which would help introduce nearby strangers to each other. Serendipity, the application created for their research, is a mobile match-making system which alerts users when someone with similar interests comes into proximity. The reactions to the system have been overwhelmingly positive. These results imply that systems which allow people to interact with familiar strangers are in fact desired by users.

\subsection{Sports}
Meeting strangers is only one part of the assumed benefits of the prototype application. Previous research suggests that doing sports in a group or together with a friend results in increased performance. Therefore, the findings suggest that finding strangers with similar interests and a similar level of fitness to do sports with would result in a performance increase for the user. However, finding people to do sports with can be an daunting task especially for people who have just moved to a new city or a country. It is important to make finding strangers as easy as possible with the use of modern technology without compromising the privacy of the users.

One of the concerns of creating the application is that how frequently people doing sports actually meet familiar strangers. Setting the level of passing by's before allowing users to communicate with each other affects the whole user experience of the prototype application. Research by \cite{runningNavigation} showed that distance is the key thing what joggers are thinking about while running, not about using familiar routes. However, while routes change, joggers use familiar locations more than once. They usually leave out a part or add one based on their overall feeling. The fact, that joggers reuse locations increases the probability of running into familiar strangers along the way.

\section{Interviews}
\subsection{Method for the interviews}
\subsection{Results}

\section{Prototype}
\subsection{Design}
\subsection{Implementation}

\section{Discussion}
\subsection{Results of the study}
\subsection{Future work}

\section{Conclusion}

% --------------------------------------------------------------------

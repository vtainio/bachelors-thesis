% Tiivistelmät tehdään viimeiseksi. 
%
% Tiivistelmä kirjoitetaan käytetyllä kielellä (JOKO suomi TAI ruotsi)
% ja HALUTESSASI myös samansisältöisenä englanniksi.
%
% Avainsanojen lista pitää merkitä main.tex-tiedoston kohtaan \KEYWORDS.

\begin{enabstract}

This thesis studies how the concept of familiar strangers, people who we meet often but don't interact with, can be integrated into a mobile application for creating social interaction among people doing sports. The literature review part of this thesis looked at the field from three perspectives. Firstly, this thesis studies social interaction among strangers by looking at various research projects. Secondly, sports and doing sports together with another person and the benefits gained from that were review as well. Lastly, the thesis looked at how social sports applications are encouraging people to do more sports and how they alter our behavior and allow new ways of doing sports with other people.

In addition to the literature review three people were interview for this thesis. The goal of the exploratory interviews ranging from 20 to 30 minutes was to get perspective and new ideas for the design details of the prototype application created for this thesis. In the end, the interviews resulted in three main ideas and findings. Firstly, there was a mixed reception for the application. Secondly, privacy was a concern for the users which lead to certain design decisions for the prototype. Lastly, the interview gave guidance towards what information should be gathered from the users and what should be public and what private.

In the and, a prototype Android application was created based on the findings from the literature review and the interviews. The application logs encounters between the user and strangers and after you have encountered the stranger for enough times, you can interact with them. The application is licended with a permissive open source license so it's possible to use it for further research or even commercial use.

\end{enabstract}

%\begin{svabstract}
%  Ett abstrakt hit 
%%(\languagename)
%\end{svabstract}

%\begin{enabstract}
% Here goes the abstract 
%%(\languagename)
%\end{enabstract}
